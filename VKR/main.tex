\documentclass[a4paper,14pt]{extarticle} % 14й шрифт
\input{include/preamble} % Подключаем преамбулу

%%% Начало документа
\begin{document}

\includepdf{include/titlepage}
%\includepdf{pz} % Пояснительная записка
\includepdf[pages=1-2]{include/task} % Задание на диплом 
\includepdf[pages=1-2]{include/referat}

\setcounter{page}{0}
\hyphenchar\font=-1 % disable hyphen
\tableofcontents % Содержание 
\hyphenchar\font=`\- % reset hyphen
\thispagestyle{empty}
\clearpage
\setcounter{page}{6} % Начало нумерации страниц

\anonsection{введение}
\input{include/intro.tex}
\section{Аналитический обзор систем стереозрения}
\input{include/intro_existing_systems.tex}            % Обзор систем
\subsection{Примеры применения сверхширокоугольных камер в робототехнике }

Рассмотрим применение и расположение сверхширокоугольных камер в некоторых существующих роботах и 
технических системах.

Камеры со сверхшироким полем зрения уже давно используются в конструкции марсоходов NASA. Типичная система 
камер ровера помимо прочего состоит из пары NavCam и трёх пар HazCam \cite{opportunity}. NavCam - это камеры, использующиеся для навигации,
 панорамных снимков и определения препятствий. Они размещены высоко, обладают узким диагональным полем зрения примерно $67^\circ$ (в марсоходе 
Perseverance повышено до $120^\circ$ \cite{perseverance}) и большим фокусным расстоянием. HazCam - это камеры, направленные на грунт перед марсоходом для обнаружения 
 препятствий в ближнем поле. Они оснащены fisheye-линзами с полем зрения примерно $120^\circ * 120^\circ$ (в марсоходе Perseverance изменены на $136^\circ * 102^\circ$).
Расположение камер изображено на рисунке \ref{pic:perseverance}. Каждая пара камер используется для построения карты глубины в соответствующей ей дальности 
и области пространства вокруг ровера. 

\addimghere{pics/perseverence}{0.7}{Расположение камер на марсоходе Perseverance; Right NavCam, Left NavCam - соответсвтенно правая и левая камера NavCam; RA и RB - правая пара камер HazCam; LA и LB - левая пара камер HazCam \ref{pic:perseverance}}{pic:perseverance}

Робот-доставщик "Ровер R3"  компании Яндекс имеет на борту 5 сверхширокоугольных камер, 
размещённые спереди, сзади и по бортам корпуса \cite{yandex_rover}. Разработчики не раскрывают их назначения, но наиболее 
вероятные варианты - визуальная одометрия и обнаружение и слежение за объектами. % FIXME: можно ли так? 
На рисунке \ref{pic:4cam_system}.а показано реальное положение сенсоров робота, а на рисунке \ref{pic:4cam_system}.б 
изображена схема зон перекрытия сверхширокоугольных камер. 

\addimghere{rover_with_zones}{0.8}{а)~положение сенсоров на роботе; б)~зоны перекрытия полей зрения}{pic:4cam_system}
 
Кроме того многие автопроизводители в своих автомобилях реализуют системы кругового обзора для помощи в парковке или 
системы автономного вождения с применением fisheye-камер. Их положение у разных производителей и моделей может отличаться, 
но общая тенденция - размещать камеры, обеспечивая максимальное покрытие окружающего пространства. Пример положения и 
углов зрения камер автомобиля Tesla представлен на рисунке \ref{pic:tesla_cams}. 

\addimghere{tesla_cameras}{0.8}{Камеры автомобиля Tesla}{pic:tesla_cams}

Рассмотренные примеры показывают системы, уже обладающие сверхширокоугольными камерами. Внедрение в них системы стереозрения 
позволило бы получить дополнительный источник информации о окружающем пространстве и/или уменьшить количество используемых камер. 
Располагаются камеры под углом, близким к $90^\circ$, что не позволяет использовать существующие методы стереозрения и 
мотивирует к разработке системы стереозрения именно для такого случая.       
%\vspace{\baselineskip}
\subsection{Выводы по первому разделу}

Обзор современных моделей искажений сверхширокоугольных камер позволил выбрать 
 ряд наиболее точных и удобных для калибровки для дальнейшего сравнений. Осуществлён
обзор существующих систем стереозрения, применяющих камеры типа "рыбий глаз" с 
целью ознакомления с мировым опытом. Принято решение разрабатывать и тестировать 
предлагаемую систему стереозрения с применением виртуального  моделирования. 

Для моделирования выбрана среда разработки Unity. Для обработки изображений
с камер выбрана библиотека OpenCV для языка программирования C++.              % Выводы по разделу 

\newpage
\section{Система стереозрения}
\subsection{Описание системы стереозрения}

Предлагаемая система стереозрения состоит из двух камер с объективами типа "рыбий глаз" $\geqslant180^\circ$,
расположенных ортогонально так, что  две камеры имеют область пересечения полей зрения. В пространстве, наблюдаемом 
сразу  двумя камерами проводится триангуляция и получение информации об объёме после этапа устранения искажений.  % FIXME: Всё описание просто бггг

% Рассмотрим организацию системы на примере робота-доставщика "Ровер R3"  компании Яндекс, который имеет на борту 4 сверхширокоугольные камеры, 
% размещённые спереди, сзади и по бортам корпуса \cite{yandex_rover}, что соответствует описанию системы. 
% На рисунке \ref{pic:4cam_system}, а показано реальное положение камер  робота и их зон перекрытия, в которых может 
%  обеспечивается получение информации о глубине при использовании описываемой системы. Рисунок \ref{pic:4cam_system}, б
% демонстрирует эквивалентную по горизонтальному  покрытию схема при использовании обычных камер 
%  с углом обзора $90^\circ$. 
 
% %\addtwoimghere{Group 1}{0.4}{Group 2}{0.4}{Сравнение систем стереозрения}{pic:4cam_system}  % TODO: разобраться с масштабом
% \addimghere{group12}{0.7}{Геометрическая модель бинокулярной системы стереозрения}{pic:4cam_system}

% Как можно заметить, системе на основе обычных камер нужно в два раза больше сенсоров, чтобы достичь той же зоны покрытия 
% по горизонтали.  Кроме того, традиционная система имеет меньшую зону покрытия по вертикали и не обеспечивает полный 
% панорамный обзор. Все эти факторы делают систему стереозрения на основе ортогонально расположенных сверхширокоугольных камер 
% более предпочтительной для применения в робототехнике.  % FIXME: уточнить. не во всей не всегда. сскорее выгодной, но  слово не очень

Применение существующих алгоритмов стереосопоставления предполагает наличие стереопары, удовлетворяющей условиям, описанным в секции \ref{stereovision}.
Такую стереопару можно получить, введя в систему  для каждой сверхширокоугольной камеры виртуальную камеру-обскуру и направив 
её в сторону пересечения полей зрения, как если бы это была часть обычной стереопары. Процесс формирования виртуальной камеры-обскуры и 
алгоритм устранения искажений более подробно описаны в секции \ref{dewarping}.

Далее для упрощения рассмотрения системы будет считаться, что оптические оси всех камер находятся в одной плоскости, 
а на ориентацию виртуальных камер влияет только угол $ \beta $ поворота в этой плоскости. 
На рисунке \ref{pic:2cam_scheme} изображён простейший вариант системы с двумя камерами под углом $90^\circ$.

\addimg{sample_simple2cam}{0.7}{Геометрическая модель бинокулярной системы стереозрения}{pic:2cam_scheme} %TODO: перерисовать схему?

  Здесь область пересечения полей зрения камер $C_0$ и $C_1$ обозначенная красным.
Эта область эквивалентна области пересечения полей зрения двух камер с полями зрения $90^\circ$ (обозначены
оранжевым), повёрнутых на $\beta_0 = \beta_1  = 45^\circ$ в сторону области интереса. Тогда $B$ - база стереопары.
Примеры исходных и желаемых изображений для каждой камеры приведён на рисунке \ref{pic:dewarped_exmples}. 

\addimghere{4pic_example}{0.7}{Пример исходных изображений и снимков виртуальной стереопары}{pic:dewarped_exmples}


      % Описание системы: Какие этапы, элементы и тд
\input{include/system_dewarping.tex}        % Как происходит выпрямление
\subsection{Выводы по второму разделу}

Описан принцип устранения искажений сверхширокоугольных линз с выбором области интереса. 
Разработан алгоритм нахождения обратной проекции для fisheye-изображения. 
Описано устройство исследуемой системы стереозрения. 

Проведённая работа позволяет приступить к моделированию системы стереозрения для проверки
заложенного принципа.         % Выводы по разделу

\newpage
\section{Моделирование системы}
\input{include/intro_virtualmodeling.tex}       % Выбор ПО  
\subsection{Виртуальное моделирование системы}

Описанная система была смоделирована в среде Unity, её внешний вид представлен на рисунке \ref{pic:unity_model}. Мир 
Unity предназначен для базовой проверки работоспособности испытываемого принципа, поэтому не содержит подробной модели
какого-либо робота. В нём присутствуют: плоскость земли, компонент, в котором сверхширокоугольные камеры закреплены под 
углом $90^\circ$, подвижный калибровочный узор и объекты-цели, предназначенные для оценки расстояния.

\addimghere{unity_view}{0.7}{Внешний вид сцены в Unity}{pic:unity_model} 

В Unity создание сцены происходит с использованием встроенных примитивов, импортированных файлов моделей популярных форматов
или моделей из магазина Asset Store, предлагающего обширную библиотеку объектов и текстур, повторяющих различные реальные объекты. В данной 
работе использовались как и примитивы для создания простых объектов, так и модели из магазина для имитации препятствий и окружения.

Для моделирования камеры <<рыбий глаз>> использовался аддон Dome Tools из магазина Unity Asset Store.    
Он позволяет моделировать сверхширокоугольные объективы с разным углом зрения в эквидистантной проекции \cite{dome_tools}. 
При этом с точки зрения других искажений, не относящихся к моделированию правильной проекции, снимки с этой камеры получаются идеальными. % FIXME: моделированию чего? 

Для виртуальной камеры доступны настройки угла зрения и виньетки по краям изображения,  % FIXME: виньетки -> абберации ??
а также различных параметров рендеринга, влияющих на уровень  детализации получаемого изображения. 

\vspace{\baselineskip}     % Моделирование системы 
% \input{include/system_code.tex}           % Про виртуальную камеру 
\subsection{Модели сверхширокоугольной камеры}
\label{camera_model}
Сложности, возникающие при использовании существующих алгоритмов стереозрения  в применении к сверхширокоугольным камерам, связаны с
 особенностями их оптической системы. Объективы  этих камер имеют в своей основе сложную систему линз, пример схемы которой вместе с примером
 получаемого изображения представлены на рисунке \ref{pic:fyscheme}. Особенности этой системы позволяют достигать очень высоких углов обзора,
  но также являются причиной аберрации и характерных искажений изображения. Чтобы описать свойства проецирования широкого набора таких
камер исследователи прибегают к аппроксимациям, называемым моделями камер. 


Модель проекции для камеры это функция, которая описывает преобразование из точки трёхмерного пространства  в области зрения 
камеры ($P=[x_c, y_c, z_c]^T$) в точку на плоскости изображения ($p=[u, \nu]^T$), как показано на \ref{pic:fy_geom}. Единичная            % не совсем единичная 
полусфера $S$ с центром в точке $O_c$ на данной схеме описывает поле зрения. На ней также лежит точка $P_C$, являющаяся результатом обратной проекции.    %$\pi^{-1}_c({p})$
Угол $\theta$ является углом падения для рассматриваемой точки, а угол $\phi$ откладывается между положительным направлением оси $x$ и $O_{i}{p}$. 

\addtwoimghere{fisheye_scheme}{0.45}{fisheye_example}{0.45}{Схема хода лучей объектива "рыбий глаз" (слева), пример изображения (справа)\cite{fy_exmp}}{pic:fyscheme}

Помимо самой проекции модели камер включают в себя описания нескольких типов искажений, накладываемых линзой. В сверхширокоугольных объективах самыми существенными являются 
радиальные - искажения, проявляющиеся сильнее по мере удаления от проекционного центра. Поэтому далее в этой секции модели будут рассматриваться именно с точки зрения 
описания радиальных искажений. Так как в этом случае искажения хода луча считаются центрально симметричными и зависят только от его удаления от центра изображения, 
большинство моделей используют координаты $\theta$ и  $r$, отмеченные на рисунке \ref{pic:fy_geom}.
% где $\lambda = \rho_c / \rho_i$ - масштабный коэффициент. 

% \addimghere{scara_graph}{0.5}{Нахождение обратной проекции для используемой модели}{pic:scara_graph} 
\addimghere{projection_geometry}{0.5}{Схема проекции точки трёхмерного пространства в точку на изображении}{pic:fy_geom}

Перспективная проекция, которая обычно используется в качестве модели ортоскопической камеры, не способна спроецировать всё широкоугольное пространство на кадр 
конечного размера. Поэтому при описании и разработке fisheye-объективов опираются на другие виды проекций  \cite{projections}:

\begin{eqseries}
    \begin{equation}
        \label{fy1}
    r = 2 f tan(\theta/2),  
    \end{equation}
    \begin{equation}
        \label{fy2}
    r = f \theta,
    \end{equation}
    \begin{equation}
        \label{fy3}
    r = 2 f sin(\theta/2),
    \end{equation}
    \begin{equation}
        \label{fy4}
    r = f sin(\theta).
    \end{equation}
\end{eqseries}    

Но реальные искажения не всегда в точности следуют заданным уравнениями.
По этой причине fisheye-проекции выгоднее аппроксимировать другими функциями \cite{opencv_model}.
В настоящий момент есть несколько распространённых моделей, аппроксимирующих реальные искажения подобных объективов. 

\subsubsection{Модель Канналы-Брандта}

Модель Канналы и Брандта \cite{opencv_model} для линз с радиально симметричными искажениями реализована в OpenCV и 
выражает их через угол падения луча света на линзу, а не расстояние                                                              \pdfmargincomment{https://stackoverflow.com/questions/31089265/what-are-the-main-references-to-the-fish-eye-camera-model-in-opencv3-0-0dev}
от центра изображения до места падения, как это делалось в более ранних моделях. Авторы посчитали, что для описания типичных искажений достаточно 
пяти членов полинома с нечётными степенями. Таким образом, указанную модель можно записать следующими уравнениями:
\begin{eqseries}
    \begin{equation}	
        \delta r = k_1\theta + k_2\theta^3 + k_3\theta^5 + k_4\theta^7 + k_5\theta^9,
        \label{eqn:kannala_r}
    \end{equation}
    \begin{equation}	
        \begin{pmatrix}u\\v\end{pmatrix} = \delta r(\theta)\begin{pmatrix}cos(\phi)\\sin(\phi)\end{pmatrix},
        \label{eqn:kannala_uv}
    \end{equation}
\end{eqseries}

где $\theta$ - угол падения луча, определяемый выбранным типом проекции,

\qquad $\phi$ - угол между горизонтом и проекцией падающего луча на плоскость изображения, 

\qquad $r = sqrt(x^2+y^2)$ - расстояние от спроектированной точки до центра изображения, 

\qquad $f$ - фокусное расстояние, 

\qquad $k_1 \dots k_5$ - параметры модели.

\subsubsection{Модель Мея}

Модель Мея \cite{mei} является более общей версией модели Гейера \cite{geyer} и позволяет использовать разные 
функции искажения для моделирования зеркал разоичного вида. Изначально она была создана для более 
эффективного моделирования катадиоптрических камер, но оказалась также весьма пригодной и для сверхширокоугольных камер. 
Калибровку этой модели можно произвести, используя библиотеку CamOdoCal. Записывается она следующим образом:

\begin{equation}
    \vspace{12pt}
    \begin{pmatrix}u\\v\end{pmatrix}=\left[\begin{array}{l}
	f_{x} \frac{x}{\alpha d+(1-\alpha) z} \\
	f_{y} \frac{y}{\alpha d+(1-\alpha) z}
	\end{array}\right]+\left[\begin{array}{l}
	c_{x} \\
	c_{y}
	\end{array}\right]
    \vspace{12pt}
\end{equation}

где $\alpha$ - параметр модели. 

\subsubsection{Модель Скарамуззы}

Также большое распространение получила модель Скарамуззы \cite{scaramuzza}, которая легла в основу Matlab Omnidirectional 
Camera Calibration Toolbox. Она связывает точки на изображении с соответствующей им точкой в координатах камеры
следующим образом:
\vskip 12pt
\begin{equation}	
    \begin{pmatrix}X_c\\Y_c\\Z_c\end{pmatrix} = \lambda \begin{pmatrix}u\\v\\a_0 + a_2 r^2 + a_3 r^3 + a_4 r^4\end{pmatrix},
    %\delta r = k_1\theta + k_2\theta^3 + k_3\theta^5 + k_4\theta^7 + ... + k_n\theta^{n+1}
    \label{eqn:scaramuzza} 
\end{equation}
\vskip 24pt

где $a_0 ... a_4$ - коэффициенты, описывающие параметры модели,

\qquad $\lambda$ - масштабный коэффициент.

Обратная проекция записывается следующим образом.
\vskip 12pt
\begin{equation}
    \label{eq:back_scara}
    \left[\begin{matrix}u_i\\v_i\\\end{matrix}\right] = \left[\begin{matrix} \frac{x_c}{\lambda}  \\  \frac{y_c}{\lambda} \\\end{matrix}\right],
\end{equation} 
\vskip 24pt
где $\lambda = \rho_c / \rho_i$. 

$\rho_i$ при этом является неизвестной. Чтобы найти её для каждого пикселя был применён метод последовательных приближений, опирающийся на прямую проекцию.
 Блок-схема алгоритма, реализующего обратную проекцию, изображена  на рисунке  \ref{pic:newton_scheme}.  Сравнение результатов выпрямления изображения инструментом  
калибровки и описанного алгоритма для центральной области изображения представлено на рисунке \ref{pic:central_pics}. 

\addimghere{flowchart}{0.5}{Блок-схема алгоритма обратной проекции}{pic:newton_scheme} 

\addimghere{remapped_images}{0.8}{Изображения, скорректированные алгоритмом (слева) и MATLAB (справа)}{pic:central_pics} 

\subsubsection{Модель Двух сфер}  % FIXME: убрать ??

Существуют и менее распространённые модели, не использующие полиномы для описания искажений. Одной из них является
модель двух сфер \cite{double_sphere}. 
Она находит положение пикселя, проектируя точку в несколько этапов - сначала на первичную сферу, потом на вторую 
сферу меньшего диаметра и смещённую на расстояние $\xi$. Наконец точка проецируется на  плоскость изображения,
 сдвинутую на расстояние $\frac{\alpha}{1-\alpha}$ относительно центра второй сферы. Модель проекции представлена 
на рисунке \ref{pic:ds_model}. Таким образом, для описания радиальных искажений достаточно всего двух параметров.
 Модель также реализована в нескольких популярные программ для калибровки камер (Basalt, Kalibr). 
 Записывается она следующим образом:
\begin{eqseries}
    \begin{equation}	
        d_1 = \sqrt{x^2+y^2+z^2}, 
    \end{equation}
    \begin{equation}	
        d_2 = \sqrt{x^2+y^2+(\xi*d_1+z)^2 }, 
    \end{equation}
    \begin{equation}	
        \begin{pmatrix}u\\v\end{pmatrix} = \begin{pmatrix}f_x * \frac{x}{\alpha*d_2+(1-\alpha)(\xi*d_1+z)} \\
                                                    f_y * \frac{y}{\alpha*d_2+(1-\alpha)(\xi*d_1+z)} \end{pmatrix}.
        %\delta r = k_1\theta + k_2\theta^3 + k_3\theta^5 + k_4\theta^7 + ... + k_n\theta^{n+1}
        \label{eqn:ds}
    \end{equation}
\end{eqseries}
\addimghere{double_sphere}{0.7}{Модель  двух сфер}{pic:ds_model}

Так как представленные модели никогда не сравнивались в пригодности для стереосопоставления, дальнейшие
эксперименты будут выполняться с применением их всех.

\vspace{\baselineskip}              % Обзор моделей камер

В этом разделе рассматриваются модели линз в их применении к описанной системе стереозрения. 
Так как классические методы стереозрения, требующие ректификации, всё ещё остаются самыми доступными и производительными \cite{disparity_review}, 
необходимо выбрать модель искажений, способную обеспечить систему наиболее точно восстановленными изображениями. 
Результатом работы алгоритма стереозрения является карта глубины воспринимаемого пространства. Подобные карты могут
использоваться, например, алгоритмами навигации и локализации для построения карты окружения робота. 
Исходные данные сравниваемых моделей представлены в приложении \ref{app-b}.

Точность работы 
стереокамер зависит от того, насколько точно карта глубины передаёт реальную информацию о форме объектов. Оценить 
эту характеристику для отдельной стереопары проблематично, так как она зависит от множества факторов. Поэтому оценка 
качества работы системы с различными моделями проведена в сравнении с  эталонами.

Кроме указанных моделей в сравнениях можно увидеть ATAN. Эта модель представляет из себя идеальную эквидистантную fisheye-проекцию, 
которая заложена в использованную виртуальную широкоугольную камеру. 
Она добавлена к сравнению как эталонный способ устранения искажений из предположения, что с ней результаты стереосопоставления будут 
самыми лучшими. Использование её на реальных объективах ограничено отсутствием возможностей по калибровке. 

Кроме того, виртуальная сцена позволяет разместить сразу несколько объектов в одной точке пространства, таким образом возможно
в существующую модель в Unity добавить ещё 2 камеры, совпадающие по параметрам и положению с виртуальными камерами на 
рисунке \ref{pic:2cam_scheme}. Разрешение новых камер выбрано исходя из размеров проекции $\nu_i$ области интереса на широкоугольном
снимке. Для камер "рыбий глаз" с разрешением $1080*1080$ пикселей она составляет 287482 пикселей, что аналогично камере 
с разрешением $540*540$. Камеры полностью подчиняются перспективной проекции (\ref{eq:uv_to_xyz}) и не содержат каких либо оптических искажений.
 Далее эти камеры будут называться эталонными камерами, а стереопара, которую они составляют - эталонной стереопарой (обозначена на графиках как REG).

Наиболее сильные искажения на изображении "рыбий глаз" возникают по краям изображения, поэтому исследуемая виртуальная стереопара построена так,
чтобы зона пересечения полей зрения камер располагалась на максимальном удалении от центра. Это соответствует ортогональному расположению
камер с углом между принципиальными осями в $90^\circ$.  При этом объективы камер имеют одинаковые параметры, поле зрения в $180^\circ$,
 а расстояние между ними составляет 20см. %Описанная конфигурация изображена на рисунке \ref{pic:2cam_scheme}.

% Здесь область пересечения полей зрения камер $C_0$ и $C_1$ обозначенная красным.
% Эта область эквивалентна области пересечения полей зрения двух камер с полями зрения $90^\circ$ (обозначены
% оранжевым), повёрнутых на $\beta_0 = \beta_1  = 45^\circ$ в сторону области интереса. Тогда $B$ - база стереопары.

\subsection{Экспериментальное исследование моделей}
\label{model_exps}

С помощью этой стереопары были получены снимки калибровочного узора шахматной доски, 
которые потом  направлены в соответствующие программные пакеты для нахождения параметров каждой модели. 
Параметры моделей занесены в разработанную библиотеку для устранения искажений на снимках виртуальной 
плоскости на расстояниях от 1 до 10м с шагом 1м. 
Пример снимка после устранения искажений и снимка с соответствующей эталонной камеры показан на рисунке \ref{pic:2images_compar}.

\addtwoimghere{pic_fy}{0.4}{pic_reg}{0.4}{Слева - снимок после устранения искажений (модель Скарамуззы); справа - эталонное изображение}{pic:2images_compar}

Визуально снимки очень похожи, однако при более детальном рассмотрении на левом изображении заметна меньшая чёткость,
вероятно, связанная с интерполяцией пикселей в процессе проецирования. Также присутствуют малозаметные искажения геометрии. Более 
явно эти дефекты для разных моделей можно увидеть на разностном изображении, представленном на рисунке \ref{pic:difference}. 

\addimghere{diffs}{1}{Разностные изображения (резкость увеличена). Чем светлее участок, тем сильнее различия}{pic:difference}

Визуальный анализ этих изображений подтверждает наблюдения и показывает малозначительность отличий для большинства моделей. 
Это позволяет применять изображения, полученные описанным путём, в стереосопоставлении. 

Оценка качества оценки глубины произведена на основе измерения среднеквадратичного отклонения и дисперсии полученных точек 
глубины от их предполагаемой позиции. Виртуальность эксперимента позволяет точно знать положение исследуемого объекта и, соответственно, 
точно определять ошибку. Как уже упоминалось, эффективность стереосопоставления \cite{SGBM} зависит от текстуры наблюдаемого объекта, 
поэтому для устранения влияния этого фактора эксперименты были проведены с различными текстурами. Использованный алгоритм стереосопоставления
опирается на поиск отклика на участок одного изображения на эпиполярной линии другого. Таким образом, текстура с одинаковыми повторяющимися 
фрагментами может привести к ложным срабатываниям и карте глубины низкого качества. Поэтому в экспериментах использовались три текстуры: часто 
повторяющийся узор обоев; текстура, состоящая как из повторяющихся с разной частотой, так и из случайных элементов; и полностью неповторяющееся изображение.
Использованные текстуры приведены на рисунке \ref{pic:textures}. Графики в этом и следующем разделе отображают усреднённые результаты по всем текстурам, отдельные графики приведены в приложениях. 

\addimghere{pics/textures}{0.7}{Пример использованных текстур}{pic:textures}

По полученным картам глубины можно осуществить 3D-реконструкция сцены. Результата реконструкции представлен  в виде облака точек, 
 аналогичного изображённому на рисунке \ref{pic:raw_pointcloud}. По соответствующим этому снимку дистанциям строится модель целевой плоскости, 
 а точки за пределами её окрестности отбрасываются. Далее из оставшихся выбираются 1000 случайных точек, что позволяет считать ошибку с 
 одинаковой точностью для всех расстояний.  % FIXME: размеры окрестности и точно ли _точность_ одинаковая? 
Эти точки используются для вычисления ошибки. 

\addimghere{pointcloud}{0.7}{Неочищенное облако точек}{pic:raw_pointcloud}

Результаты оценки качества нахождения глубины приведены на рисунке \ref{pic:quality}. По оси абсцисс отложено расстояние до исследуемого объекта,
по оси ординат среднеквадратичное отклонение. На рисунке \ref{pic:mean} изображено зависимость матожидания дистанции до поверхности от расстояния 
до исследуемой плоскости и дисперсия точек.  Каждая ломанная соответствует своей модели и усреднена по всем текстурам.

\addimghere{pics/depth_quality}{1}{Точность построения карты глубины}{pic:quality}

\addimghere{pics/distance}{1}{Оценка дистанции до плоскости}{pic:mean}

Как можно заметить по графикам, наилучший результат в оценке глубины логичным образом показывает эталонная стереопара. Далее следует 
идеальная модель, заложенная в виртуальную камеру, демонстрируя, что даже в случае полного соответствие прямой и обратной проекций часть 
информации в изображении теряется или искажается. Из исследуемых моделей самую низкую ошибку демонстрирует модель Канналы-Брандта с результатом
0.06м на метр удаления, у других моделей ошибка выше и достигает до 0.1м на метр удаления. Кроме того заметно, что большинство моделей имеют тенденцию
к недооценке расстояния.  Дисперсия на максимальном расстоянии варьируется от $\pm 0.0096$ у эталонной стереопары и $\pm 0.0763$ у модели Канналы-Брандта 
до $\pm 0.393$ у модели Мея. % TODO: актуализировать




\subsection{Выводы по третьему разделу}

Выбрано ПО для разработки и виртуально моделирования. Разработана виртуальная модель системы и тестовая среда в Unity. 
Отобраны несколько распространённых моделей сверхширокоугольных камер и выполнена их калибровка.
Исследованы изображения, полученные с помощью виртуальных камер поле устранения искажений. Выполнено их сравнение 
с эталонными изображениями,  которое продемонстрировало пригодность снимков для стереосопоставления.

Проведён эксперимент
по оценке качества облака точек, полученного с помощью предлагаемого решения, в сравнении с традиционной системой. 
Он показал, что в применении к стереозрению наилучшие результаты среди рассмотренных демонстрирует модель Канналы и Брандта. 
Она показывает наименьший прирост ошибки при данных условиях, который составляет в среднем 0.06м на метр удаления от камеры, 
что, однако, на 27\% больше, чем у эталонной стереопары. Кроме того, модель является довольно доступной в плане автоматической 
калибровки и присутствует в нескольких библиотеках технического зрения. Несмотря на заметную разницу в точности на симулированных 
данных, в реальном применении она может сократиться из-за наличия искажений и у "обычных" камер. Изучению системы 
 с применением реальных камер посвящена следующая глава. 


% \newpage
% \section{Исследование точности калибровочных моделей сверхширокоугольных объективов}

\newpage
\section{Экспериментальное исследование системы стереозрения}

Предлагаемая стереосистема показала свою работоспособность и удовлетворительные результаты в виртуальных тестах 
точности. Это позволяет перейти к испытаниям системы на реальных камерах. 

Для этого использованы 2 имеющиеся в наличии камеры % TODO: написать тип + ссылки
с линзами 1.45mm F2.2 1/1.8 FOV $190^\circ$ (AC123B0145IRM12MM), закреплённые под углом $90^\circ$ в корпусе,
полученным 3D-печатью. База стереопары составляет в таком случае примерно  $72мм$. Изображение модуля системы 
стереозрения представлен на рисунке \ref{pic:cam_case}. 

\addimghere{pics/case_photo}{1}{Система стереозрения в корпусе}{pic:cam_case}

Пример, пары fisheye-изображений, выдаваемых таким модулем камер представлен на рисунке \ref{pic:stere_fy_img}. 
Как можно заметить, у этих камер круг линзы не полностью вписан в кадр, а её центр смещён относительно центра кадра,
 что уменьшает количество полезной площади кадра, доступной для устранения искажений. Кроме того, заметна виньетка 
по краям, которая может мешать поиску соответствий. Всё это повышает вклад сенсоров в общую ошибку. 

\addimghere{pics/real_fy_shots}{1}{Снимки с двух реальных камер с объективами "рыбий глаз"}{pic:stereo_fy_img}

Ошибка условий измерений минимизируется использованием искусственного освещения, расстояние до целевого объекта
контролируется разметкой на экспериментальном столе, нанесённой с помощью рулетки. Аналогично виртуальным испытаниям 
использованы 3 текстуры, распечатанные на листе матовой бумаги, для уменьшения влияния свойств наблюдаемой поверхности.

Каждая камера по-отдельности откалибрована по снимкам узора шахматной доски, полученные параметры параметры моделей 
представлены в таблице \ref{}. Примеры изображений с устранёнными искажениями для каждой модели представлены на рисунке
 \ref{pic:dewarped_exmples_real}. Заметно что на них присутствуют пустоты, вызванные усечённым форматом оригинального 
 изображения. Тем не менее, эти пустоты не мешают дальнейшему эксперименту, так как оставляют достаточное поле зрения 
 незатронутым.

\addimg{pics/dewarped_real}{1}{Примеры изображений с устранёнными искажениями}{pic:dewarped_exmples_real} 
\vspace{\baselineskip}
\subsection{Экспериментальное исследование реальной системы стереозрения}

Экспериментальная установка состоит из модуля системы стереозрения, закреплённого на краю стола, 
разметки с шагом 25см и целевого объекта в виде коробки с приклеенным изображением текстуры. Снимок 
экспериментальной установки представлен на рисунке \ref{pic:table_photo}. 

\addimghere{pics/table}{0.8}{Экспериментальная установка}{pic:table_photo}

С помощью этой установки получены снимки целевого объекта с тремя разными текстурами, вручную помещённого на
удаление от 0.25м до 1.5м с шагом 0.25м. Они переданы в тот же алгоритм подсчёта ошибки, описанный в \ref{model_exps}. 

Результаты оценки качества нахождения глубины приведены на рисунке \ref{pic:quality_real}. По оси абсцисс 
отложено расстояние до исследуемого объекта, по оси ординат среднеквадратичное отклонение. На графике так же 
представлена кривая, соответствующая теоретической точности традиционной стереопары с параметрами аналогичными 
исследуемой виртуальной стереопаре. На рисунке \ref{pic:mean_real}
 изображено зависимость матожидания дистанции до поверхности от расстояния до исследуемой плоскости и дисперсия точек.
Каждая ломанная соответствует своей модели и усреднена по всем текстурам.

\addimghere{pics/depth_quality_real}{1}{Точность построения карты глубины}{pic:quality_real}

\addimghere{pics/distance_real}{1}{Оценка дистанции до плоскости}{pic:mean_real}

Как можно заметить по графикам, наименьшая ошибка возникает у модели Канналы-Брандта и достигает примерно 10\%  на удалении в метр, 
у моделей Мея и Скарамуззы она составляет на том же расстоянии примерно 12\%. Это в среднем в 3 раза большая среднеквадратичная 
ошибка по сравнению с результатами виртуальных экспериментов и теоретической точности. Кроме того заметно, что все модели в этом эксперименте имеют тенденцию
к переоценке расстояния.  

Наблюдения, сделанные о характере распределения точек по глубине в виртуальном эксперименте, справедливы и для 
реальной стереопары.



% Рассмотрим работу модуля кругового стереозрения на основе четырёх камер, аналогичного 
% изображённому на рисунке \ref{pic:ssvs}. Применение описанного ранее алгоритма позволяет
% использовать каждую камер сразу в двух стереопарах, обеспечивая широкое покрытие.
% Подобный модуль мог бы обеспечить робота как полным панорманым обзором, так и почти полным панорамной оценкой глубины. 

% Сборка виртуальной модели системы произведена в Unity. Модуль помещён в одну из локаций 
% и использован для получения снимков со всех камер. По полученным картам глубины построено
% общее облако точек, изображённое вместе с исходной локацией на рисунке \ref{pic:module_cloud}. 

% \addimghere{unity_with_ptc}{1}{Слева - локация, в которыой были сделаны снимки; справа - результат реконструкции (обрезан по высоте)}{pic:module_cloud}
% % \addtwoimghere{unity_topdown}{0.4}{white_ptc}{0.4}{Слева - локация, в которыой были сделаны снимки; справа - результат реконструкции (обрезан по высоте)}{pic:module_cloud}

% В облаке заметны очертания окружающих поверхностей. Дальнейшие испытания целесообразно проводить 
% с использованием одного из распространённых пакетов автономной локализации и построения карты в рамках.
\vspace{\baselineskip}
\subsection{Заключение по пятому разделу}
% Описанная ранее виртуальная модель системы стереозрения позволяет проводить 
с ней испытания для оценки работоспособности и сравнения с аналогами в контролируемой среде.           

Результатом работы алгоритма стереозрения является карта глубины воспринимаемого пространства. Подобные карты могут
использоваться, например, алгоритмами навигации и локализации для построения карты окружения робота. Точность работы 
этих алгоритмов зависит от того, насколько точно карта глубины передаёт реальную информацию о форме объектов. Оценить 
эту характеристику для отдельной системы стереозрения проблематично, так как она зависит от множества факторов.         % TODO: дописать факторы или сослаться на статью
Поэтому оценка качества работы системы проведена в сравнении с виртуальной моделью традиционной стереопары.

Здесь в виртуальных экспериментах для сравнения также применяется эталонная стереопара.  % TODO: чувствуется какая-то недосказанность
       % План экспериментов
% \input{include/experiments_pic.tex}         % Сравнение картинок
% \input{include/experiments_ptc.tex}         % Оценка равномерности поинтклауда стены
% \input{include/experiments_conclusion.tex}  % Выводы по разделу

\newpage
\anonsection{Заключение}
\input{include/conclusion_conclusion.tex} %Заключение-заключение

\newpage
\anonsection{СПИСОК ИСПОЛЬЗОВАННЫХ ИСТОЧНИКОВ}
\renewcommand{\refname}{}
\patchcmd{\thebibliography}{\section*{\refname}}{}{}{}
\patchcmd{\thebibliography}{\addcontentsline{toc}{section}{\refname}}{}{}{}
\patchcmd{\thebibliography}{\begin{itemize}}{\begin{itemize}[leftmargin=30mm]}{}{}
    \makeatletter
    \renewcommand*{\@biblabel}[1]{\hfill#1}
    \makeatother
{\def\section*#1{}
\bibliography{refs.bib} 
}
\newpage
% Приложения
\appsection{Приложение А} \hypertarget{app-a}{\label{app-a}}

\centering{Сравнение ПО для симуляции}


    \begin{table}[h!]             % TODO: Сыровато. Надо дополнить. Проверить ГОСТовость подписи  FIXME: Таблица не умещается, но вращать не хочется. 
        
        %\caption{Сравнение ПО для симуляции }
        \label{tab:sims}
        \rotatebox{90} {\begin{tabular}{|l|l|l|l|l|}
        \hline
        \textbf{Название} & \textbf{\begin{tabular}[c]{@{}l@{}}Симуляция \\ fisheye-камер\end{tabular}} & \textbf{\begin{tabular}[c]{@{}l@{}}Реалистичное \\ моделирование\end{tabular}} & \textbf{Интеграция кода}                                            & \textbf{Доступность}  \\ \hline
        Gazebo            & Возможна                                                                    & Возможно                                                                     & \begin{tabular}[c]{@{}l@{}}Возможна \\ посредством ROS\end{tabular} & Бесплатно               \\ \hline
        RoboDK            & Нет                                                                         & Затруднено                                                                     & Нет                                                                 & От 145€                \\ \hline
        Webots            & Затруднена                                                                  & Возможно                                                                       & Возможна                                                            & Бесплатно              \\ \hline
        CoppeliaSim       & Затруднена                                                                  & Затруднено                                                                     & Возможна                                                            & Бесплатно              \\ \hline
        NVIDIA Isaac Sim  & Возможна                                                                    & Возможно                                                                       & Возможна                                                            & Бесплатно              \\ \hline
        CARLA             & Затруднена                                                                  & Возможно                                                                       & Возможна                                                            & Бесплатно              \\ \hline
        Unity             & Возможна                                                                    & Возможно                                                                       & Возможна                                                            & Бесплатно              \\ \hline
        \end{tabular}}

    \end{table}
\clearpage

% \appsection{Приложение Б} \hypertarget{app-a}{\label{app-b}}

% \centering{Исходные данные моделей}

% \begin{table}[h!]
%     % \caption{Исходные данные}
%     \label{tab:models_params}
%     \rotatebox{90} {\begin{tabular}{|l|l|l|l|}
%         \hline
%     Модель         & ПО калибровки & \begin{tabular}[c]{@{}l@{}}Ошибка\\ калибровки\end{tabular}   & Параметры модели \\ \hline
%     Мей            & OdoCamCalib   & 0.102               & \begin{tabular}[c]{@{}l@{}} $  \xi=1.47431, p_1=0.000166, p_2=0.00008,$ \\ $k_1=-0.208916, k_2=0.153247 $                        \end{tabular}           \\ \hline
%     Каннала-Брандт & OdoCamCalib   & 0.099               & \begin{tabular}[c]{@{}l@{}} $  k_2=0.000757676, k_3=-0.000325907, k_4=0.0000403,$ \\ $k_5=-0.000001866, f_x=343.57, f_y=343.37$  \end{tabular}   \\ \hline
%     Скарамузза     & MATLAB        & 0.121               & \begin{tabular}[c]{@{}l@{}} $  a_0=345.1319, a_1=-0.0011, a_2=5.7623 * 10^-7,$ \\$ a_3=-1.3985 * 10^-9 $                         \end{tabular}      \\ \hline
%     \end{tabular} }
% \end{table}
% \clearpage

% \appsection{Приложение Б} \hypertarget{app-b}{\label{app-b}}

% \centering{Программный код файла FisheyeDewarper.cpp}
% \begin{english}
% \lstinputlisting[language=C++, numbers=left]{code/FisheyeDewarper.cpp}
% \end{english}

% \clearpage

% \appsection{Приложение В} \hypertarget{app-c}{\label{app-c}}

% \centering{Программный код файла unity\_plugin.cpp}
% \begin{english}
% \lstinputlisting[language=C++, numbers=left]{code/unity_plugin.cpp}
% \end{english}

% \clearpage

% \appsection{Приложение Г} \hypertarget{app-d}{\label{app-d}}

% \centering{Программный код файла Connector.cs}
% \begin{english}
% \lstinputlisting[language=C, numbers=left]{code/Connector.cs}
% \end{english}

% \clearpage

% \appsection{Приложение Д} \hypertarget{app:matlab}{\label{app:matlab}}

% \centering{Программный код файла DisparityPlayground.m}
% \begin{english}
% \lstinputlisting[language=Matlab, numbers=left]{code/DisparityPlayground.m}
% \end{english}

% \clearpage
 % Код скрипта

% \lof
% \lot

\end{document} 
%%% Конец документа
