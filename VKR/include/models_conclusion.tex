\subsection{Выводы по третьему разделу}

Выбрано ПО для разработки и виртуально моделирования. Разработана виртуальная модель системы и тестовая среда в Unity. 
Отобраны несколько распространённых моделей сверхширокоугольных камер и выполнена их калибровка.
Исследованы изображения, полученные с помощью виртуальных камер поле устранения искажений. Выполнено их сравнение 
с эталонными изображениями,  которое продемонстрировало пригодность снимков для стереосопоставления.

Проведён эксперимент
по оценке качества облака точек, полученного с помощью предлагаемого решения, в сравнении с традиционной системой. 
Он показал, что в применении к стереозрению наилучшие результаты среди рассмотренных демонстрирует модель Канналы и Брандта. 
Она показывает наименьший прирост ошибки при данных условиях, который составляет в среднем 0.06м на метр удаления от камеры, 
что, однако, на 27\% больше, чем у эталонной стереопары. Кроме того, модель является довольно доступной в плане автоматической 
калибровки и присутствует в нескольких библиотеках технического зрения. Несмотря на заметную разницу в точности на симулированных 
данных, в реальном применении она может сократиться из-за наличия искажений и у "обычных" камер. Изучению системы 
 с применением реальных камер посвящена следующая глава. 
