
\subsection{Стереозрение}
\label{stereovision}
Система стереозрения состоит из двух камер, наблюдающих сцену с разных точек. 
Фундаментальная основа принципа заключается в предположении, что каждой точке в пространстве соответствует уникальная пара пикселей на снимках с двух камер.  

При этом к камерам предъявляются некоторые требования \cite{rusoverview}:   % не уверен, что это надо цитировать
\begin{itemize}
	\item Камеры откалиброваны. Это значит, что известны внутренние (оптические) и внешние (расположение камер в пространстве) параметры камер. 
	\item Ректификация. Подразумевает выравнивание изображения с обеих камер по строкам.  % Мб подробнее расписать  
	\item Ламбертовость поверхностей. Означает независимость освещения наблюдаемых поверхностей от угла зрения. 
\end{itemize}

Таким образом, соблюдение указанных выше требований позволяет использовать следующий геометрический принцип. При наличии двух камеры, как изображено 
на рисунке \ref{pic:epipol} \cite{Hartley2004}, где $C$ — центр первой камеры, $C'$ — центр второй камеры, точка пространства $X$  
проецируется в точки $x$ на плоскости изображения левой камеры и $x'$ на плоскости изображения правой камеры. Прообразом точки $x$ на изображении левой 
камеры является луч $xX$. Этот луч проецируется на плоскость второй камеры в прямую $l'$, называемую эпиполярной линией. Образ точки $X$ на плоскости 
изображения второй камеры обязательно лежит на этой линии.

\addimghere{epipolar geometry}{0.5}{Эпиполярная геометрия}{pic:epipol}

В результате каждой точке $x$ на изображении левой камеры соответствует эпиполярная линия $l'$ на изображении правой камеры. При этом соответствие для $x$ на 
изображении правой камеры может лежать только на соответствующей эпиполярной линии. Аналогично, каждой точке $x'$ на правом изображении соответствует 
эпиполярная линия $l$ на левом. Поиск соответствий между точками $x$ и $x'$ называют стереосопставлением.

Далее с помощью точек $x$ и $x'$ возможно посчитать смещения каждого пикселя одного изображения относительно другого, что позволяет построить карту смещений. 
Очевидно, что смещения будут подсчитаны только для точек, видимых обеими камерами. Карта смещений же приводится далее либо к облаку точек, либо к карте глубины. 
Стереосистемы, реализующие этот принцип, называют пассивными. Они являются самыми простыми и часто используются, так как для их создания достаточно лишь
двух зафиксированных камер. Однако пассивные системы опираются целиком на видимый свет, что ограничивает их применимость в экстремальных условиях освещённости. %\cite{find:passive_perfomance}. 

Одну из камер можно заменить источником  света, освещающим одну или несколько точек поверхности световым лучом или специальным структурированным шаблоном освещения.  % TODO: цитирование страницы  529
Так как структура шаблона и направление его  лучей заранее известны, камера может проводить оценку формы объектов в кадре по искажениям шаблона\cite{shapiro}. 
Структурная подсветка тоже является распространённым методом, реализованным во многих коммерческих сенсорах, благодаря совмещению высокой производительности 
и  низкой цены \cite{struct_light}.  

Совмещение пассивного стереозрения и структурированной подсветки позволяет улучшить качество стереосопоставления, особенно для поверхностей со слабо выраженной текстурой
и при низкой освещённости. Однако инфракрасный узор подсветки весьма ограничен в дальности и не виден при сильном солнечном освещении, что затрудняет использование этих    % FIXME: криво ппц
сенсоров на открытых пространствах. Системы с подобной технологий называют активными \cite{kinect_perf}. 

\addimghere{stereo_types}{1.0}{Виды организации стереосистем: а)~пассивная стереосистема; б)~активная стереосистема; в)~подсветка структирированным светом }{pic:stereo_methods}
%\pdfmargincomment{просится какой-то вывод}

Описанные подходы хорошо работают для обычных камер с незначительными искажениями, где легко выполнить ректификацию. 		%\pdfmargincomment{и сюда сразу модель обскуры или потом оставить?}
В случае  же камер с широкоугольными (от 60° до 100°), особоширокоугольными (от 100° до 180°) и сверхширокоугольными (свыше 180°)  \cite{camera_class} объективами
 в изображение вносятся существенные искажения, которые затрудняют поиск соответствий. На 
рисунке \ref{pic:fy_epipol} вручную сопоставлены одни и те же точки на фрагментах двух изображений со сверхширокоугольных камер,
смотрящих в одном направлении. Видно, что пары признаков больше не лежат на одной горизонтальной прямой.  
% Учитывая распространённость таких  камер в робототехнике, задача реализации систем стереозрения на их основе является актуальной.    

\addimghere{sample}{0.7}{Соответствия на снимках с fisheye-камер}{pic:fy_epipol}   % TODO: без поворота

Существует несколько основных направлений исследований в области методов стереозрения, применимых к таким камерам:
\begin{itemize}
\item исследования стереопар с двумя камерами со сверхширокоугольными объективами, направленными в одну сторону и имеющими параллельные оптические оси \cite{parallel_fy, sweep_net};
\item исследования стереопар с двумя камерами со сверхширокоугольными объективами, направленными в противоположные стороны и имеющими коллинеарные оптические оси;
\item исследования стереопар с двумя камерами со сверхширокоугольными объективами, имеющие ортогонально ориентированными оптические оси, лежащие в одной плоскости \cite{ortho_fy};
\end{itemize}

Первый подход является одним из самых распространённых. Последние работы в этой области предлагают \cite{direct_neuro_stereo} отказаться от типичных 
для стереозрения этапов устранения искажений и ректификации и извлекать информацию о глубине по двум снимкам сверхширокоугольных камер без 
предварительной обработки. Для производства карт глубины используется свёрточная неиронная сеть, что требует существенных вычислительных 
мощностей - для достижения производительности в реальном времени разработчикам понадобилось использовать настольный компьютер c мощным графическим ускорителем. 
В мобильном автономном роботе аналогичный по производительности вычислитель разместить может быть проблематично. 

С помощью второго подхода может быть реализована, например, кольцевая область стереозрения с вертикальным  %Метод с разворотом камер на 180  ...
полем зрения $65^\circ$ \cite{omni_stereo}. Для этого две $245^\circ$ камеры закреплены на противоположных концах жёсткого стержня 
и направлены друг на друга. Это позволяет достигнуть панорамного обзора глубины с качеством, достаточным для осуществления автономной навигации и
локализации БПЛА\cite{omni_copter}, но конструктивно такая схема расположения камер имеет смысл только для летательных аппаратов.  

Третий подход пока самый малоисследованный. В рамках него Чжан (Zhang) и другие разработали особую систему стереозрения (the special stereo vision 
system), использующую модуль из четырёх камер с полем зрения $185^\circ$. Размещены камеры в одной плоскости под углом в $90^\circ$ друг к другу, 
как изображено на рисунке \ref{pic:ssvs} \cite{zhang_system}. Авторы отразили в работе калибровку разработанной ими системы и устранение искажений, 
но не  продемонстрировали результаты оценки глубины и не проанализировали точность метода. 

\addimghere{ssvs}{0.6}{Модуль камер}{pic:ssvs}
