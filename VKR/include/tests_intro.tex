
Предлагаемая стереосистема показала свою работоспособность и удовлетворительные результаты в виртуальных тестах 
точности. Это позволяет перейти к испытаниям системы на реальных камерах. 

Для этого использованы 2 имеющиеся в наличии камеры % TODO: написать тип + ссылки
с линзами 1.45mm F2.2 1/1.8 FOV $190^\circ$ (AC123B0145IRM12MM), закреплённые под углом $90^\circ$ в корпусе,
полученным 3D-печатью. База стереопары составляет в таком случае примерно  $72мм$. Изображение модуля системы 
стереозрения представлен на рисунке \ref{pic:cam_case}. 

Пример, пары fisheye-изображений, выдаваемых таким модулем камер представлен на рисунке \ref{pic:stereo_fy_img}. 
Как можно заметить, у этих камер круг линзы не полностью вписан в кадр, а её центр смещён относительно центра кадра,
 что уменьшает количество полезной площади кадра, доступной для устранения искажений. Кроме того, заметна виньетка 
по краям, которая может мешать поиску соответствий. Всё это повышает вклад сенсоров в общую ошибку. 

\addimghere{pics/case_photo}{0.6}{Система стереозрения в корпусе}{pic:cam_case}


\addimghere{pics/real_fy_phot}{0.8}{Снимки с реальной камеры с объективом "рыбий глаз"}{pic:stereo_fy_img}

Ошибка условий измерений минимизируется использованием искусственного освещения, расстояние до целевого объекта
контролируется разметкой на экспериментальном столе, нанесённой с помощью рулетки. Аналогично виртуальным испытаниям 
использованы 3 текстуры, распечатанные на листе матовой бумаги, для уменьшения влияния свойств наблюдаемой поверхности.

Каждая камера по-отдельности откалибрована по снимкам узора шахматной доски, полученные параметры параметры моделей 
представлены в таблице \ref{tab:models_params}. Примеры изображений с устранёнными искажениями для каждой модели представлены на рисунке
 \ref{pic:dewarped_exmples_real}. Заметно что на них присутствуют пустоты, вызванные усечённым форматом оригинального 
 изображения. Тем не менее, эти пустоты не мешают дальнейшему эксперименту, так как оставляют достаточное поле зрения 
 незатронутым.

 \begin{table}[h!]
    
    \caption{Результаты калибровки}
    \label{tab:models_params}
    \resizebox{\textwidth}{!}{%
    \begin{tabular}{|l|l|l|l|}
        \hline
    \multicolumn{1}{|c|}{Модель}  & \multicolumn{1}{|c|}{ПО калибровки} & \multicolumn{1}{|c|}{Ошибка}   & \multicolumn{1}{|c|}{Параметры модели} \\ \hline
    Мей            & CamOdoCal   & 0.102               & \begin{tabular}[c]{@{}l@{}} $  \xi=1.47431, p_1=0.000166, p_2=0.00008,$ \\ $k_1=-0.208916, k_2=0.153247 $                        \end{tabular}           \\ \hline
    Каннала-Брандт & CamOdoCal   & 0.099               & \begin{tabular}[c]{@{}l@{}} $  k_2=0.000757676, k_3=-0.000325907, k_4=0.0000403,$ \\ $k_5=-0.000001866, f_x=343.57, f_y=343.37$  \end{tabular}   \\ \hline
    Скарамузза     & MATLAB        & 0.121               & \begin{tabular}[c]{@{}l@{}} $  a_0=345.1319, a_1=-0.0011, a_2=5.7623 * 10^-7,$ \\$ a_3=-1.3985 * 10^-9 $                         \end{tabular}      \\ \hline
    \end{tabular} }
\end{table}

\addimg{pics/dewarped_real}{1}{Примеры изображений с устранёнными искажениями}{pic:dewarped_exmples_real}