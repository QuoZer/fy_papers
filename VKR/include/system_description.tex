\subsection{Описание системы стереозрения}

Предлагаемая система стереозрения состоит из двух камер с объективами типа "рыбий глаз" $\geqslant180^\circ$,
расположенных ортогонально так, что  две камеры имеют область пересечения полей зрения. В пространстве, наблюдаемом 
сразу  двумя камерами проводится триангуляция и получение информации об объёме после этапа устранения искажений.  % FIXME: Всё описание просто бггг

% Рассмотрим организацию системы на примере робота-доставщика "Ровер R3"  компании Яндекс, который имеет на борту 4 сверхширокоугольные камеры, 
% размещённые спереди, сзади и по бортам корпуса \cite{yandex_rover}, что соответствует описанию системы. 
% На рисунке \ref{pic:4cam_system}, а показано реальное положение камер  робота и их зон перекрытия, в которых может 
%  обеспечивается получение информации о глубине при использовании описываемой системы. Рисунок \ref{pic:4cam_system}, б
% демонстрирует эквивалентную по горизонтальному  покрытию схема при использовании обычных камер 
%  с углом обзора $90^\circ$. 
 
% %\addtwoimghere{Group 1}{0.4}{Group 2}{0.4}{Сравнение систем стереозрения}{pic:4cam_system}  % TODO: разобраться с масштабом
% \addimghere{group12}{0.7}{Геометрическая модель бинокулярной системы стереозрения}{pic:4cam_system}

% Как можно заметить, системе на основе обычных камер нужно в два раза больше сенсоров, чтобы достичь той же зоны покрытия 
% по горизонтали.  Кроме того, традиционная система имеет меньшую зону покрытия по вертикали и не обеспечивает полный 
% панорамный обзор. Все эти факторы делают систему стереозрения на основе ортогонально расположенных сверхширокоугольных камер 
% более предпочтительной для применения в робототехнике.  % FIXME: уточнить. не во всей не всегда. сскорее выгодной, но  слово не очень

Применение существующих алгоритмов стереосопоставления предполагает наличие стереопары, удовлетворяющей условиям, описанным в секции \ref{stereovision}.
Такую стереопару можно получить, введя в систему  для каждой сверхширокоугольной камеры виртуальную камеру-обскуру и направив 
её в сторону пересечения полей зрения, как если бы это была часть обычной стереопары. Процесс формирования виртуальной камеры-обскуры и 
алгоритм устранения искажений более подробно описаны в секции \ref{dewarping}.

Далее для упрощения рассмотрения системы будет считаться, что оптические оси всех камер находятся в одной плоскости, 
а на ориентацию виртуальных камер влияет только угол $ \beta $ поворота в этой плоскости. 
На рисунке \ref{pic:2cam_scheme} изображён простейший вариант системы с двумя камерами под углом $90^\circ$.

\addimg{sample_simple2cam}{0.7}{Геометрическая модель бинокулярной системы стереозрения}{pic:2cam_scheme} %TODO: перерисовать схему?

  Здесь область пересечения полей зрения камер $C_0$ и $C_1$ обозначенная красным.
Эта область эквивалентна области пересечения полей зрения двух камер с полями зрения $90^\circ$ (обозначены
оранжевым), повёрнутых на $\beta_0 = \beta_1  = 45^\circ$ в сторону области интереса. Тогда $B$ - база стереопары.
Примеры исходных и желаемых изображений для каждой камеры приведён на рисунке \ref{pic:dewarped_exmples}. 

\addimghere{4pic_example}{0.7}{Пример исходных изображений и снимков виртуальной стереопары}{pic:dewarped_exmples}


