\subsection{Экспериментальное исследование реальной системы стереозрения}

Экспериментальная установка состоит из модуля системы стереозрения, закреплённого на краю стола, 
разметки с шагом 25см и целевого объекта в виде коробки с приклеенным изображением текстуры. Снимок 
экспериментальной установки представлен на рисунке \ref{pic:table_photo}. 

\addimghere{pics/table}{0.8}{Экспериментальная установка}{pic:table_photo}

С помощью этой установки получены снимки целевого объекта с тремя разными текстурами, вручную помещённого на
удаление от 0.25м до 1.5м с шагом 0.25м. Они переданы в тот же алгоритм подсчёта ошибки, описанный в \ref{model_exps}. 

Результаты оценки качества нахождения глубины приведены на рисунке \ref{pic:quality_real}. По оси абсцисс 
отложено расстояние до исследуемого объекта, по оси ординат среднеквадратичное отклонение. На графике так же 
представлена кривая, соответствующая теоретической точности традиционной стереопары с параметрами аналогичными 
исследуемой виртуальной стереопаре. На рисунке \ref{pic:mean_real}
 изображено зависимость матожидания дистанции до поверхности от расстояния до исследуемой плоскости и дисперсия точек.
Каждая ломанная соответствует своей модели и усреднена по всем текстурам.

\addimghere{pics/depth_quality_real}{1}{Точность построения карты глубины}{pic:quality_real}

\addimghere{pics/distance_real}{1}{Оценка дистанции до плоскости}{pic:mean_real}

Как можно заметить по графикам, наименьшая ошибка возникает у модели Канналы-Брандта и достигает примерно 10\%  на удалении в метр, 
у моделей Мея и Скарамуззы она составляет на том же расстоянии примерно 12\%. Это в среднем в 3 раза большая среднеквадратичная 
ошибка по сравнению с результатами виртуальных экспериментов и теоретической точности. Кроме того заметно, что все модели в этом эксперименте имеют тенденцию
к переоценке расстояния.  

Наблюдения, сделанные о характере распределения точек по глубине в виртуальном эксперименте, справедливы и для 
реальной стереопары.



% Рассмотрим работу модуля кругового стереозрения на основе четырёх камер, аналогичного 
% изображённому на рисунке \ref{pic:ssvs}. Применение описанного ранее алгоритма позволяет
% использовать каждую камер сразу в двух стереопарах, обеспечивая широкое покрытие.
% Подобный модуль мог бы обеспечить робота как полным панорманым обзором, так и почти полным панорамной оценкой глубины. 

% Сборка виртуальной модели системы произведена в Unity. Модуль помещён в одну из локаций 
% и использован для получения снимков со всех камер. По полученным картам глубины построено
% общее облако точек, изображённое вместе с исходной локацией на рисунке \ref{pic:module_cloud}. 

% \addimghere{unity_with_ptc}{1}{Слева - локация, в которыой были сделаны снимки; справа - результат реконструкции (обрезан по высоте)}{pic:module_cloud}
% % \addtwoimghere{unity_topdown}{0.4}{white_ptc}{0.4}{Слева - локация, в которыой были сделаны снимки; справа - результат реконструкции (обрезан по высоте)}{pic:module_cloud}

% В облаке заметны очертания окружающих поверхностей. Дальнейшие испытания целесообразно проводить 
% с использованием одного из распространённых пакетов автономной локализации и построения карты в рамках.