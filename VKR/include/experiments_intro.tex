Описанная ранее виртуальная модель системы стереозрения позволяет проводить 
с ней испытания для оценки работоспособности и сравнения с аналогами в контролируемой среде.           

Результатом работы алгоритма стереозрения является карта глубины воспринимаемого пространства. Подобные карты могут
использоваться, например, алгоритмами навигации и локализации для построения карты окружения робота. Точность работы 
этих алгоритмов зависит от того, насколько точно карта глубины передаёт реальную информацию о форме объектов. Оценить 
эту характеристику для отдельной системы стереозрения проблематично, так как она зависит от множества факторов.         % TODO: дописать факторы или сослаться на статью
Поэтому оценка качества работы системы проведена в сравнении с виртуальной моделью традиционной стереопары.

Здесь в виртуальных экспериментах для сравнения также применяется эталонная стереопара.  % TODO: чувствуется какая-то недосказанность
