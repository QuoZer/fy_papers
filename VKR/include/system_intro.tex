\subsection{Примеры применения сверхширокоугольных камер в робототехнике }

Рассмотрим применение и расположение сверхширокоугольных камер в некоторых существующих роботах и 
технических системах.

Камеры со сверхшироким полем зрения уже давно используются в конструкции марсоходов NASA. Типичная система 
камер ровера помимо прочего состоит из пары NavCam и трёх пар HazCam \cite{opportunity}. NavCam - это камеры, использующиеся для навигации,
 панорамных снимков и определения препятствий. Они размещены высоко, обладают узким диагональным полем зрения примерно $67^\circ$ (в марсоходе 
Perseverance повышено до $120^\circ$ \cite{perseverance}) и большим фокусным расстоянием. HazCam - это камеры, направленные на грунт перед марсоходом для обнаружения 
 препятствий в ближнем поле. Они оснащены fisheye-линзами с полем зрения примерно $120^\circ * 120^\circ$ (в марсоходе Perseverance изменены на $136^\circ * 102^\circ$).
Расположение камер изображено на рисунке \ref{pic:perseverance}. Каждая пара камер используется для построения карты глубины в соответствующей ей дальности 
и области пространства вокруг ровера. 

\addimghere{pics/perseverence}{0.7}{Расположение камер на марсоходе Perseverance; Right NavCam, Left NavCam - соответсвтенно правая и левая камера NavCam; RA и RB - правая пара камер HazCam; LA и LB - левая пара камер HazCam \ref{pic:perseverance}}{pic:perseverance}

Робот-доставщик "Ровер R3"  компании Яндекс имеет на борту 5 сверхширокоугольных камер, 
размещённые спереди, сзади и по бортам корпуса \cite{yandex_rover}. Разработчики не раскрывают их назначения, но наиболее 
вероятные варианты - визуальная одометрия и обнаружение и слежение за объектами. % FIXME: можно ли так? 
На рисунке \ref{pic:4cam_system}.а показано реальное положение сенсоров робота, а на рисунке \ref{pic:4cam_system}.б 
изображена схема зон перекрытия сверхширокоугольных камер. 

\addimghere{rover_with_zones}{0.8}{а)~положение сенсоров на роботе; б)~зоны перекрытия полей зрения}{pic:4cam_system}
 
Кроме того многие автопроизводители в своих автомобилях реализуют системы кругового обзора для помощи в парковке или 
системы автономного вождения с применением fisheye-камер. Их положение у разных производителей и моделей может отличаться, 
но общая тенденция - размещать камеры, обеспечивая максимальное покрытие окружающего пространства. Пример положения и 
углов зрения камер автомобиля Tesla представлен на рисунке \ref{pic:tesla_cams}. 

\addimghere{tesla_cameras}{0.8}{Камеры автомобиля Tesla}{pic:tesla_cams}

Рассмотренные примеры показывают системы, уже обладающие сверхширокоугольными камерами. Внедрение в них системы стереозрения 
позволило бы получить дополнительный источник информации о окружающем пространстве и/или уменьшить количество используемых камер. 
Располагаются камеры под углом, близким к $90^\circ$, что не позволяет использовать существующие методы стереозрения и 
мотивирует к разработке системы стереозрения именно для такого случая. 