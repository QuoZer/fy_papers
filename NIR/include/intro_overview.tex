\subsection{Модели сверхширокоугольной камеры}
\label{camera_model}
Сложности, возникающие при использовании существующих алгоритмов стереозрения  в применении к сверхширокоугольным камерам, связаны с
 особенностями их оптической системы. Объективы  этих камер имеют в своей основе сложную систему линз, схема которой вместе с примером
 получаемого изображения представлена на рисунке \ref{pic:fyscheme}. Особенности этой системы позволяют достигать углов обзора свыше $180^\circ$,
  но также являются причиной аберрации и характерных искажений изображения. Реалистично моделировать ход лучей в подобных камерах 
нецелесообразно, поэтому исследователи прибегают к аппроксимациям, называемым моделями камер.   % FIXME: meh

\addtwoimghere{fisheye_scheme}{0.45}{fisheye_example}{0.45}{Схема хода лучей объектива "рыбий глаз" (слева), пример изображения (справа)\cite{fy_exmp}}{pic:fyscheme}

Как видно по рисунку \ref{pic:fy_geom}, модель проекции для камеры это функция, которая моделирует преобразование 
из точки трёхмерного пространства  в области зрения камеры ($P=[x_c, y_c, z_c]^T$) в точку на плоскости изображения ($p=[u, \nu]^T$). Единичная            % не совсем единичная 
полусфера $S$ с центром в точке $O_c$ описывает поле зрения. На ней также лежит точка $P_C$, являющаяся результатом обратной проекции.    %$\pi^{-1}_c({p})$
Угол $\theta$ является углом падения для рассматриваемой точки, а угол $\phi$ откладывается между положительным направлением оси $x$ и $O_{i}{p}$. 


Модели камер включают в себя описания нескольких типов искажений, накладываемых линзой, но в сверхширокоугольных объективах самыми существенными являются 
радиальные - искажения, проявляющиеся сильнее ближе к краям изображения. Поэтому далее в этой секции модели будут рассматриваться именно с точки зрения 
описания радиальных искажений. 

\addimghere{projection_geometry}{0.5}{Схема проекции точки трёхмерного пространства в точку на изображении}{pic:fy_geom}

Перспективная проекция, которая обычно используется в качестве модели ортоскопической камеры, не способна спроецировать широкоугольное пространство на кадр 
конечного размера. Поэтому при описании и разработке fisheye-объективов опираются на другие виды проекций  \cite{projections}: 
\begin{equation}
    \label{fy1}
r = 2 f tan(\theta/2),  
\end{equation}
\begin{equation}
    \label{fy2}
r = f \theta,
\end{equation}
\begin{equation}
    \label{fy3}
r = 2 f sin(\theta/2),
\end{equation}
\begin{equation}
    \label{fy4}
r = f sin(\theta),
\end{equation}

Но реальные линзы не всегда в точности следуют заданным моделям, к тому же отличия в используемых параметрах усложняют процесс калибровки камер. 
По этой причине радиальные искажения выгоднее аппроксимировать многочленами \cite{opencv_model}, например, вида
 \begin{equation}	
	\begin{split}
        \delta r= k_1 r^3 + k_2 r^5 + k_3 r^7 + ... + k_n r^{n+2},
        \label{eqn:fisheye_distortion}
    \end{split}
\end{equation}
где $\delta r$ - радиальное отклонение от идеальной проекции луча;  $k_i$ - коэффициенты, описывающие внутренние параметры камеры. 

В настоящий момент есть несколько распространённых моделей, аппроксимирующих реальные искажения подобных объективов. Модель Канналы и 
Брандта \cite{opencv_model} для линз с радиально симметричными искажениями реализована в OpenCV и выражает их через угол падения луча света на линзу, а не расстояние  \pdfmargincomment{https://stackoverflow.com/questions/31089265/what-are-the-main-references-to-the-fish-eye-camera-model-in-opencv3-0-0dev}
от центра изображения до места падения, как это делалось в более ранних моделях. Авторы посчитали, что для описания типичных искажений достаточно 
пяти членов полинома. Таким образом, указанную модель можно записать следующими уравнениями:

\begin{equation}	
    \delta r = k_1\theta + k_2\theta^3 + k_3\theta^5 + k_4\theta^7 + k_5\theta^9,
    \label{eqn:kannala_r}
\end{equation}

\begin{equation}	
    \begin{pmatrix}u\\v\end{pmatrix} = \delta r(\theta)\begin{pmatrix}cos(\phi)\\sin(\phi)\end{pmatrix},
    %\delta r = k_1\theta + k_2\theta^3 + k_3\theta^5 + k_4\theta^7 + ... + k_n\theta^{n+1}
    \label{eqn:kannala_uv}
\end{equation}
где $\theta$ - угол падения луча, определяемый выбранным типом проекции; $\phi$ - угол между горизонтом 
и проекцией падающего луча на плоскость изображения; $r = sqrt(x^2+y^2)$ - расстояние от спроектированной точки до центра
изображения; $f$ - фокусное расстояние.

Также большое распространение получила модель Скарамуззы \cite{scaramuzza}, которая легла в основу Matlab Omnidirectional 
Camera Calibration Toolbox. Она связывает точки на изображении с соответствующей им точкой в координатах камеры фьук
следующим образом
\begin{equation}	
    \begin{pmatrix}X_c\\Y_c\\Z_c\end{pmatrix} = \lambda \begin{pmatrix}u\\v\\a_0 + a_2 r^2 + a_3 r^3 + a_4 r^4\end{pmatrix},
    %\delta r = k_1\theta + k_2\theta^3 + k_3\theta^5 + k_4\theta^7 + ... + k_n\theta^{n+1}
    \label{eqn:scaramuzza}
\end{equation}
где $a_0 ... a_4$ - коэффициенты, описывающие внутренние параметры камеры; $\lambda$ - масштабный коэффициент.

В качестве модели fisheye-камеры в этой работе принята именно модель Скарамуззы, так как она обладает точностью на уровне 
других \cite{double_sphere} и удобным инструментом оценки внутренних параметров камеры.

% Уменьшить  число параметров калибровки (принять $a_1$) в этой модели позволяет наблюдение 

Существуют и менее распространённые модели, не использующие полиномы для описания искажений. Одной из них является
модель двух сфер \cite{double_sphere}. % TODO: перевести получше
Она положение пикселя, проектируя точку сначала на первичную сферу, потом на вторую сферу поменьше, сдвинутую на 
расстояние $\xi$, а затем в плоскость изображения камеры-обскуры, сдвинутой на $\frac{\alpha}{1-\alpha}$ относительно 
центра второй сферы. Модель проекции представлена на рисунке \ref{pic:ds_model}. Таким образом радиальные искажения 
можно описать всего двумя параметрами. Модель также интегрирована
во многие популярные программы для калибровки камер (Basalt, Kalibr).

\addimghere{double_sphere}{0.7}{Модель  двух сфер}{pic:ds_model}



 