\include{settings}

\begin{document} % начало документа
\raggedbottom
%\include{titlepage}


% Содержание
%\tableofcontents
%\newpage
\title{Сравнение качества построения карты глубины при использовании различных калибровочных моделей сверхширокоугольных объективов}
\author{Пантелеев М.}
\institute{Санкт-Петербургский Политехнический Университет Петра Великого
\email{panteleev.md@edu.spbstu.ru}}
\maketitle

\begin{abstract}
	% % TODO: дополнить  
	% В этой статье приводится обзор основных алгоритмов, связанных с получением глубины по снимкам одной сцены с нескольких камер. Улучшение существующих и
	% создание новых методов происходит постоянно, так как проблема точного и быстрого стерео до сих пор не решена. Разные подходы склоняются либо в сторону высокой, 
	% производительности, либо в сторону повышенной точности. Алгоритмы, способные выполнять сопоставление в реальном времени представляют особый интерес для 
	% исследователей, поэтому выбранные методы рассматриваются также на возможность такого применения.

\end{abstract}

\section{Введение}

Применения не ограничивается роботами, переписать.

За последние годы был достигнут существенный прогресс в доступности и точности сенсоров, позволяющих мобильным роботам 
проводить оценку окружающего пространства. Такие информационно-измерительные устройства как лидары, сонары и стереокамеры
 стали основным источником информации для алгоритмов автономной навигации и локализации. Тем не менее в роботах по-прежнему 
присутствуют телевизионные системы, так как они дают наиболее легко воспринимаемую информацию для оператора в случаях, когда 
его вмешательство необходимо. "Обычные" камеры, описываемые перспективной проекцией дают изображение, понятное для 
восприятия и обработки, но покрывают зачастую слишком маленькую область пространства. Когда же нужно большее поле зрения, 
могут быть использованы катадиоптрические системы, состоящие из выгнутого зеркала и перспективной камеры. Однако такой метод 
не всегда применим, так как система получается громоздкой и имеет "мёртвую зону" посередине кадра. Наконец, можно использовать 
камеры "рыбий глаз" (англ. fisheye-camera), позволяющие с помощью специальной системы линз одним кадром покрыть угловое 
поле свыше $180^\circ$. В сравнении с катадиоптрическими они обладают большей полезной областью кадра. %% можно ли так говорить? 

\begin{figure}[H]
	\begin{center}
		\includegraphics[scale=0.5]{pics/sample.png}
		\caption{Сравнение изображений с разных типов камер} 
		\label{pic:epipol} % название для ссылок внутри кода
	\end{center}
\end{figure}

При этом, как видно по изображению \ref{pic:epipol} (в), сверхшироукольные объективы накладывают на изображение заметные искажения.
Обычно принято рассматривать два вида искажений: тангенциальные и радиальные, но в данном случае тангенциальные принебрежимы по сравнению
 с радиальными и далее рассматриваться не будут. Устранение этих искажений является важной задачей, так как её решение позволяет 
 нивилировать недостатки сверхширокоугольных объективов и применять их для чувствительных к точности передачи формы объектов в кадре задач.

 Одной из таких задач является стереозрение. Несмотря на существование методов, позволяющих оценивать глубину по полным fisheye-снимкам \cite{full_fisheye_stereo}, 
классические методы, требующие ректификации, всё ещё остаются самыми доступными и производительными. Построив систему стереозрения на 
основе ортогонально расположенных сверхширокоугольных камер, можно получить ряд преимуществ перед традиционными камерами, но для этого 
сначала нужно выбрать модель искажений, наиболее точно описывающую данные линзы. [здесь бы сослаться на самого себя, но пока не на что,
и тогда, получается, что принцип этой системы тоже надо бы описать].

Оптическая система была смоделирована в виртуальной среде с применением игрового движка Unity и плагина ZybrVR Dome Tools. %% TODO: что-то ещё про виртуальный опыт 

\subsection{Модели камер "рыбий глаз"}

Сверхширокоугольные линзы изготавливают, закладывая разные виды проекций, их примеры приведены на рисунке \ref{pic:projections}, но 
но реальные линзы не всегда в точности соответствуют им, поэтому для более точного описания принято использовать модели
камер на основе других функций.
Определим основные обозначения. Модель проекции для камеры это функция (обычно обозначаемая $\pi_c(\cdot )$), которая моделирует преобразование 
из точки трёхмерного пространства ($P=[x_c, y_c, z_c]^T$) в поле зрения камеры в точку на плоскости изображения ($p=[u, \nu]^T$). Тогда обратная
проекция - это ... Единичная            % не совсем единичная 
полусфера $S$ с центром в точке $O_c$ описывает поле зрения. На ней также лежит точка $P_C$, являющаяся результатом обратной проекции $\pi^{-1}_c({p})$.
Угол $\theta$ является углом падения для рассматриваемой точки, а угол $\phi$ откладывается между положительным направлением оси $u$ и $O_{i}{p}$. 
% TODO: переписать текст выше более правильным способом
\addimghere{pics/projection_geometry}{0.5}{Схема проекции точки трёхмерного пространства в точку на изображении}{pic:fy_geom}

Далее будут рассмотрены основные модели камер отобранные для сравнения.

\subsubsection{KB}

\subsubsection{Scaramuzza}

\subsubsection{Atan}

\subsection{Сравнение качества построения карт глубины}



\section{Выводы}
\label{conclusion}


\newpage
\bibliographystyle{./config/splncs04}
\bibliography{refs}

\end{document}
