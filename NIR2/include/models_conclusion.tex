\subsection{Выводы по третьему разделу}

Исследованы изображения, полученные с помощью виртуальных камер поле устранения искажений. Выполнено их сравнение 
с эталонными изображениями,  которое продемонстрировало пригодность снимков для использования в системах стереозрения.

Проведён эксперимент,
по оценке качества облака точек, полученного с помощью предлагаемого решения, в сравнении  традиционной системой. Он показал, что в применении к стереозрению 
наилучшие результаты среди рассмотренных демонстрирует модель Канналы и Брандта. 
Она поазывает наименьший прирост ошибки при данных условиях, который составляет в среднем 0.06м на метр удаления от камеры, что, однако, на 27\% больше, 
чем у эталонной стереопары. Кроме того, модель является довольно доступной в плане автоматической калибровки и присутствует в нескольких библиотеках 
технического зрения. Несмотря на заметную разницу в точности на симулированных данных, в реальном применении она может сократиться из-за наличия искажений
у "обычных" камер. Изучению системы в том числе с применением реальных камер посвящена следующая глава. 
% Полученные результаты сравнимы с актуальными пассивными система стереозрения \cite{passive_perf}, что позволяет перейти
% к дальнейшим этапам реализации системы стереозрения на основе ортогонально расположенных камер.
В этот раздел планируется добавление ещё одного эксперимента по оценке степени "скашивания" облаков точек и ещё одной модели, качество калибровки которой пока
неудовлетворительно. 