\subsection{Выводы по третьему разделу}

Проведённые исследования показали, что в применении к стереозрению наилучшие результаты демонстрирует модель Канналы и Брандта. Она демонстрирует наименьший 
прирост ошибки, который составляет в среднем 0.06м на метр удаления от камеры, что на 27\% больше, чем для эталонной стереопары. Кроме того, модель является довольно доступной 
в плане автоматической калибровки и доступной в нескольких библиотеках технического зрения. Несмотря на меньшую точность, 
стереосистема, основанная на камерах со сверхширокоугольными объективами, может успешно применяться \cite{}.

В этот раздел планируется добавление ещё одного эксперимента по оценке степени "скашивания" облаков точек. 