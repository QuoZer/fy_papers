\subsection{Тестирование панорамной системы стереозрения}

Рассмотрим работу модуля кругового стереозрения на основе четырёх камер, аналогичного 
изображённому на рисунке \ref{pic:ssvs}. Применение описанного ранее алгоритма позволяет
использовать каждую камер сразу в двух стереопарах, обеспечивая широкое покрытие.
Подобный модуль мог бы обеспечить робота как полным панорманым обзором, так и почти полным панорамной оценкой глубины. 

Сборка виртуальной модели системы произведена в Unity. Модуль помещён в одну из локаций 
и использован для получения снимков со всех камер. По полученным картам глубины построено
общее облако точек, изображённое вместе с исходной локацией на рисунке \ref{pic:module_cloud}. 

\addimghere{unity_with_ptc}{1}{Слева - локация, в которыой были сделаны снимки; справа - результат реконструкции (обрезан по высоте)}{pic:module_cloud}
% \addtwoimghere{unity_topdown}{0.4}{white_ptc}{0.4}{Слева - локация, в которыой были сделаны снимки; справа - результат реконструкции (обрезан по высоте)}{pic:module_cloud}

В облаке заметны очертания окружающих поверхностей. Дальнейшие испытания целесообразно проводить 
с использованием одного из распространённых пакетов автономной локализации и построения карты в рамках.