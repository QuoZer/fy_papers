\subsection{Виртуальное моделирование системы}


Описанная система была смоделирована в среде Unity, её внешний вид представлен на рисунке \ref{pic:unity_model}. Мир 
Unity предназначен для базовой проверки работоспособности испытываемого принципа, поэтому не содержит подробной модели
какого-либо робота. В нём присутствуют: плоскость земли, кронштейн, на котором сверхширокоугольные камеры закреплены под 
углом $90^\circ$, подвижный калибровочный узор и объекты-цели, предназначенные для оценки расстояния. В силу особенностей 
работы многих алгоритмов стереозрения эти объекты должны сильнотекстурированы \cite{disparity_review}. 

\addimghere{unity_view}{0.7}{Внешний вид сцены в Unity}{pic:unity_model} 

В Unity создание сцены происходит с использованием встроенных примитивов, импортированных файлов моделей популярных форматов
или моделей из магазина ассетов, предлагающего обширную библиотеку объектов и текстур, повторяющих различные реальные объекты. В данной 
работе использовались как и примитивы для создания простых объектов, так и модели из магазина для имитации препятствий и окружения.

Для моделирования камеры "рыбий глаз" использовался аддон Dome Tools из магазина Unity Asset Store.    
Он позволяет моделировать сверхширокоугольные объективы с разным углом зрения в эквидистантной проекции \cite{dome_tools}. 
При этом с точки зрения других искажений, не относящихся к моделированию правильной проекции, снимки с этой камеры получаются идеальными. % FIXME: моделированию чего? 
Окно настроек камеры представлено на рисунке \ref{pic:camera_settings}.

\addimghere{camera_settings}{0.5}{Окно настроек виртуальной камеры в Unity}{pic:camera_settings} 

Для виртуальной камеры доступны настройки угла зрения и виньетки по краям изображения,  % FIXME: виньетки -> абберации ??
а также различных параметров рендеринга, влияющих на уровень  детализации получаемого изображения. 