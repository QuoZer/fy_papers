
\subsection{Стереозрение}
\label{stereovision}
Система стереозрения состоит из двух камер, наблюдающих сцену с разных точек, как изображено на рисунке \ref{pic:epipol} \cite{Hartley2004}. 
Фундаментальная основа принципа заключается в предположении, что каждой точке в пространстве соответствует уникальная пара пикселей на снимках с двух камер.  

При этом к камерам предъявляются некоторые требования \cite{rusoverview}:   % не уверен, что это надо цитировать
\begin{itemize}
	\item Камеры откалиброваны. Это значит, что известны внутренние (оптические) и внешние (расположение камер в пространстве) параметры камер. 
	\item Ректификация. Подразумевает выравнивание изображения с обеих камер по строкам.  % Мб подробнее расписать  
	\item Ламбертовость поверхностей. Означает независимость освещения наблюдаемых поверхностей от угла зрения. 
\end{itemize}

Таким образом, соблюдение указанных выше требований позволяет использовать следующий геометрический принцип. При наличии двух камеры, как изображено 
на рисунке \ref{pic:epipol}, где $C$ — центр первой камеры, $C'$ — центр второй камеры, точка пространства $X$  
проецируется в $x$ на плоскость изображения левой камеры и в $x'$ на плоскость изображения правой камеры. Прообразом точки $x$ на изображении левой 
камеры является луч $xX$. Этот луч проецируется на плоскость второй камеры в прямую $l'$, называемую эпиполярной линией. Образ точки $X$ на плоскости 
изображения второй камеры обязательно лежит на эпиполярной линии $l'$.

\addimghere{epipolar geometry}{0.5}{Эпиполярная геометрия}{pic:epipol}

В результате каждой точке $x$ на изображении левой камеры соответствует эпиполярная линия $l'$ на изображении правой камеры. При этом соответствие для $x$ на 
изображении правой камеры может лежать только на соответствующей эпиполярной линии. Аналогично, каждой точке $x'$ на правом изображении соответствует 
эпиполярная линия $l$ на левом. 

% TODO: стересопоставление

Далее с помощью точек $x$ и $x'$ возможно посчитать смещения каждого пикселя одного изображения относительно другого, что позволяет построить карту смещений. 
Очевидно, что смещения будут подсчитаны только для точек, видимых обеими камерами. Карта смещений же приводится далее либо к облаку точек, либо к карте глубины. 
Стереосистемы, реализующие этот принцип, называют пассивными. Они являются самыми простыми и часто используются, так как для их изготовления достаточно 
двух зафиксированных камер. Однако пассивные системы опираются целиком на видимый свет, что ограничивает их применимость в условиях с малой освещённостью. %\cite{find:passive_perfomance}. 

Одну из камер можно заменить источником  света, освещающим одну или несколько точек поверхности световым лучом или специальным шаблоном освещения.  % TODO: цитирование страницы  529
Так как структура шаблона и направление его  лучей заранее известны, камера может проводить оценку формы объектов в кадре по искажениям шаблона\cite{shapiro}. 
Структурная подсветка тоже является распространённым методом, реализованным во многих коммерческих сенсорах,  благодаря совмещению высокой производительности 
и  низкой цены \cite{struct_light}.  

Совмещение пассивного стереозрения и структурированной подсветки позволяет улучшить качество стереосопоставления, особенно для поверхностей со слабо выраженной текстурой
и при низкой освещённости. Однако инфракрасный узор подсветки весьма ограничен в дальности и не виден при сильном солнечном освещении, что затрудняет использование этих    % FIXME: криво ппц
сенсоров на открытых пространствах. Системы с подобной технологий называют активными \cite{kinect_perf}. 

\addimghere{stereo_types}{1.0}{Виды организации стереосистем: а) пассивная стереосистема; б) активная стереосистема; в) подсветка структирированным светом }{pic:fy_epipol}
\pdfmargincomment{просится какой-то вывод}

Описанные подходы хорошо работают для обычных камер с незначительными искажениями, где легко выполнить ректификацию. 		\pdfmargincomment{и сюда сразу модель обскуры или потом оставить?}
В случае  же камер со сверхшироким углом обзора в изображение вносятся существенные искажения, которые затрудняют поиск соответствий. На 
рисунке \ref{pic:fy_epipol} вручную сопоставлены одни и те же точки на фрагментах двух изображений. Видно, что пары признаков больше не лежат
на одной горизонтальной прямой.  Учитывая распространённость таких  камер в робототехнике, задача реализации систем стереозрения на их основе 
является актуальной.    % Вообще какое-то введение опять, действительно ли распространены

\addimghere{not_epipolar}{0.8}{Соответствия на снимках с fisheye-камер}{pic:fy_epipol}   % TODO: найти/нарисовать

Исследователи предложили несколько реализаций систем стереозрения, опирающихся на снимки со сверхширокоугольных камер. 
Например, с помощью  пары таких камер реализована  кольцевая область стереозрения с вертикальным  %Метод с разворотом камер на 180  ...
полем зрения $65^\circ$\cite{omni_stereo}. Для этого две $245^\circ$ камеры закреплены на противоположных концах жёсткого стержня 
и направлены друг на друга.  
Это позволяет достигнуть панорамного обзора глубины с качеством, достаточным для осуществления автономной навигации и
локализации \cite{omni_copter}, но конструктивно такая схема расположения камер имеет смысл только для летательных аппаратов.  

Рохас и Оиши \cite{direct_neuro_stereo} решили отказаться от типичных для стереозрения этапов устранения искажений и ректификации % FIXME: пояснить ректификацию? 
и извлекать информацию о глубине напрямую по двум снимкам fisheye-камер. Для производства карт глубины используется 
свёрточная неиронная сеть, что требует существенных вычислительных мощностей - для достижения производительности в реальном 
времени разработчикам понадобилось использовать компьютер с ЦПУ i7-4770 и ГПУ NVIDIA GTX 1080Ti. В мобильном автономном роботе
аналогичный по производительности вычислитель разместить может быть проблематично. Кроме того, метод реализован лишь для случая, когда
 обе камеры направлены в одном направлении. 

Были предложены и ортогональные системы, использующие большие площади перекрытия полей зрения объективов "рыбий глаз" для 
стереозрения. % FIXME: фиксить формулировку, некрасиво
Чжан (Zhang) и другие разработали особую систему стереозрения (the special stereo vision system), использующую модуль из четырёх камер с углом зрения
$185^\circ \times 185^\circ$. Размещёны камеры в одной плоскости под углом в $90^\circ$, как изображено на рисунке \ref{pic:ssvs} \cite{zhang_system}.
Авторы отразили в работе калибровку разработанной ими системы и устранение искажений, но не проанализировали точность метода. Кроме того,
в работе не продемонстрированы  результаты  оценки глубины. 

\addimghere{ssvs}{0.7}{Модуль камер}{pic:ssvs}

% Распространённые библиотеки машинного зрения (OpenCV, MATLAB CV Toolbox) предлагаю готовые к использованию классы и функции, позволяющие после калибровки
% камер получать с помощью них карты глубины. Однако на практике эти методы весьма ограничены. Для стереосопоставления 
% используются традиционные методы, приспособленные для классических камер с перспективной проекцией, что не позволяет 
% использовать кадры с широкоугольных камер целиком. В результате у этих кадров после устранения искажений остаётся 
% угол зрения не более $120^\circ$. Кроме того, библиотечные функции не позволяют задать область интереса для каждой камеры, 
% что ограничивает их область применения только для копланарного расположения сенсоров.  