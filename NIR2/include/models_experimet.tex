

С помощью этой стереопары были получены снимки калибровочного узора шахматной доски, 
которые потом  направлены в соответствующие программные пакеты для нахождения параметров каждой модели. Далее с помощью алгоритма, описанного в \ref{},				% TODO: сделать нормальную ссылку
выполнялось устранение искажений для набора изображений, собранного виртуальными камерами. По полученной паре изображений 
можно проводить стереосопоставление.

Оценка качества оценки глубины произведена на основе измерения среднеквадратичного отклонения и дисперсии полученных точек 
глубины от их предполагаемой позиции. Виртуальность эксперимента позволяет точно знать положение исследуемого объекта и, соответственно, 
точно определять ошибку. Кроме того, это позволяет изменять геометрические размеры исследуемого объекта в процессе исследования, чтобы его
угловые размеры в поле зрения камеры оставались постоянными. Таким образом возможно получить карты глубины примерно одинаковой 
плоскости на всём диапазоне дистанций.  Эффективность стереосопоставления \cite{SGBM} сильно зависит от текстуры наблюдаемого объекта, 
поэтому для устранения влияния этого фактора эксперименты были проведены с различными текстурами. 
Примеры использованных текстур приведены на рисунке \ref{pic:textures}. 

\addimghere{pics/textures}{0.7}{Пример использованных текстур}{pic:textures}

Результаты оценки качества нахождения глубины приведены на рисунке \ref{pic:quality}. По оси абсцисс отложено расстояние до исследуемого объекта,
по оси ординат среднеквадратичное отклонение. На рисунке \ref{pic:mean} изображено матожидание расстояния до исследуемой плоскости 
и дисперсия точек. По оси абсцисс отложено реальное расстояние, а по оси ординат - матожидание. Каждый график соответствует своей модели.

Как можно заметить по графику, наилучший результат в оценке глубины логичным образом показывает эталонная стереопара. Далее следует 
идеальная модель, заложенная в виртуальную камеру, демонстрируя, что даже в случае идеального соответствие прямой и обратной проекции часть 
информации в изображении теряется или искажается. Из исследуемых моделей самую низкую ошибку демонстрирует модель Канналы-Брандта с результатом
0.06м на метр удаления. 

