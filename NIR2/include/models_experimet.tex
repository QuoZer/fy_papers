\subsection{Экспериментальное исследование моделей}

С помощью этой стереопары были получены снимки калибровочного узора шахматной доски, 
которые потом  направлены в соответствующие программные пакеты для нахождения параметров каждой модели. Далее с помощью алгоритма, 
описанного в главе \ref{dewarping},		
выполнено устранение искажений для набора изображений с целевой плоскостью, собранного виртуальными камерами. 

Снимки после устранения искажений и с эталонной камеры показаны на рисунке \ref{pic:2images_compar}.

\addtwoimghere{pic_fy}{0.4}{pic_reg}{0.4}{Слева - снимок после устранения искажений (модель Скарамуззы); справа - эталонное изображение}{pic:2images_compar}

Визуально снимки очень похожи, однако при более детальном рассмотрении на левом изображении заметна меньшая чёткость
из-за интерполяции в процессе проецирования. Также присутствуют малозаметные искажения геометрии. Более 
явно эти дефекты для разных моделей можно увидеть на разностном изображении, представленном на рисунке \ref{pic:difference}. 

\addimghere{diffs}{1}{Разностные изображения (резкость увеличена). Чем светлее участок, тем сильнее различия}{pic:difference}

Визуальный анализ этих изображений подтверждает наблюдения и показывает малозначительность отличий для большинства моделей. Это позволяет применять
изображения, полученные таким путём, в стереосопоставлении. 

Оценка качества оценки глубины произведена на основе измерения среднеквадратичного отклонения и дисперсии полученных точек 
глубины от их предполагаемой позиции. Виртуальность эксперимента позволяет точно знать положение исследуемого объекта и, соответственно, 
точно определять ошибку. Кроме того, это позволяет изменять геометрические размеры исследуемого объекта в процессе исследования, чтобы его
угловые размеры в поле зрения камеры оставались постоянными. Таким образом возможно получить карты глубины примерно одинаковой 
плоскости на всём диапазоне дистанций.  Эффективность стереосопоставления \cite{SGBM} сильно зависит от текстуры наблюдаемого объекта, 
поэтому для устранения влияния этого фактора эксперименты были проведены с различными текстурами. 
Примеры использованных текстур приведены на рисунке \ref{pic:textures}. 

\addimghere{pics/textures}{0.7}{Пример использованных текстур}{pic:textures}

Построение карты расхождений происходит методом полу-глобального  сопоставления, после которого осуществляется 3D-реконструкция сцены. 
Результата реконструкции представлен  в виде облака точек,  изображённого на рисунке \ref{pic:raw_pointcloud}.
По заданным для данного снимка параметрам строится модель целевой плоскости, а точки за пределами её окрестности отбрасываются. % FIXME: размеры окрестности
Оставшиеся точки используются для вычисления ошибки. 
 
\addimghere{pointcloud}{0.7}{Неочищенное облако точек}{pic:raw_pointcloud}

Результаты оценки качества нахождения глубины приведены на рисунке \ref{pic:quality}. По оси абсцисс отложено расстояние до исследуемого объекта,
по оси ординат среднеквадратичное отклонение. На рисунке \ref{pic:mean} изображено зависимость матожидания дистанции до поверхности от расстояния 
до исследуемой плоскости и дисперсия точек. По оси абсцисс отложено реальное расстояние, а по оси ординат - матожидание. Каждый график соответствует своей модели.

\addimghere{pics/depth_quality}{1}{Точность построения карты глубины}{pic:quality}

\addimghere{pics/distance}{1}{Оценка дистанции до плоскости}{pic:mean}

Как можно заметить по графику, наилучший результат в оценке глубины логичным образом показывает эталонная стереопара. Далее следует 
идеальная модель, заложенная в виртуальную камеру, демонстрируя, что даже в случае идеального соответствие прямой и обратной проекций часть 
информации в изображении теряется или искажается. Из исследуемых моделей самую низкую ошибку демонстрирует модель Канналы-Брандта с результатом
0.06м на метр удаления, у других моделей ошибка выше и достигает 0.1м на метр удаления. Кроме того заметно, что большинство моделей имеют тенденцию
к недооценке расстояния.  Дисперсия на максимальном расстоянии варьируется от $\pm 0.0096$ у эталонной стереопары и $\pm 0.0763$ у модели Канналы-Брандта 
до $\pm 0.393$ у модели Мея. 



