
В этом разделе рассматриваются описанные модели линз в их применении к стереозрению. 
Так как классические методы стереозрения, требующие ректификации, всё ещё остаются самыми доступными и производительными \cite{disparity_review}, 
необходимо выбрать модель искажений, способную обеспечить систему наиболее точно восстановленными изображениями. 

Исходные данные сравниваемых моделей представлены в таблице \ref{tab:models_params}.

\begin{table}[]
    \centering
    \caption{Исходные данные}
    \label{tab:models_params}
    \begin{tabular}{llll}
    Модель         & ПО калибровки & MRE   & Параметры модели \\
    Мей            & OdoCamCalib   & 0.102 &                  \\
    Каннала-Брандт & OdoCamCalib   & 0.099 & Коэффициенты     \\
    Скарамузза     & MATLAB        & 0.121 & Коэффициенты    
    \end{tabular}
\end{table}

Кроме указанных моделей в сравнениях можно увидеть ATAN. Эта модель представляет из себя идеальную эквидистантную fisheye-проекцию, 
которая заложена в использованную виртуальную широкоугольную камеру. 
Она добавлена к сравнению как эталонный способ устранения искажений из предположения, что с ней результаты стереосопоставления будут 
самыми лучшими.  
Кроме этой проекции для сравнения в экспериментах присутствуют две идеальные камеры без искажений, направленные
ровно в том же направлении, в котором будет производиться устранение  искажений для всех остальных моделей.  

Наиболее сильные искажения на изображении "рыбий глаз" возникают по краям изображения, поэтому исследуемая виртуальная стереопара построена так,
чтобы зона пересечения полей зрения камер располагалась на максимальном удалении от центра. Это соответствует ортогональному расположению
камер с углом между принципиальными осями в $90^\circ$.  При этом объективы камер имеют одинаковые параметры, поле зрения в $180^\circ$,
 а расстояние между ними составляет 20см. %Описанная конфигурация изображена на рисунке \ref{pic:2cam_scheme}.

% Здесь область пересечения полей зрения камер $C_0$ и $C_1$ обозначенная красным.
% Эта область эквивалентна области пересечения полей зрения двух камер с полями зрения $90^\circ$ (обозначены
% оранжевым), повёрнутых на $\beta_0 = \beta_1  = 45^\circ$ в сторону области интереса. Тогда $B$ - база стереопары.
