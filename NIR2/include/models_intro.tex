
В этом разделе рассматриваются модели линз в их применении к описанной системе стереозрения. 
Так как классические методы стереозрения, требующие ректификации, всё ещё остаются самыми доступными и производительными \cite{disparity_review}, 
необходимо выбрать модель искажений, способную обеспечить систему наиболее точно восстановленными изображениями. 
Результатом работы алгоритма стереозрения является карта глубины воспринимаемого пространства. Подобные карты могут
использоваться, например, алгоритмами навигации и локализации для построения карты окружения робота. 
Исходные данные сравниваемых моделей представлены в приложении \ref{app-b}.

Точность работы 
стереокамер зависит от того, насколько точно карта глубины передаёт реальную информацию о форме объектов. Оценить 
эту характеристику для отдельной стереопары проблематично, так как она зависит от множества факторов. Поэтому оценка 
качества работы системы с различными моделями проведена в сравнении с  эталонами.

Кроме указанных моделей в сравнениях можно увидеть ATAN. Эта модель представляет из себя идеальную эквидистантную fisheye-проекцию, 
которая заложена в использованную виртуальную широкоугольную камеру. 
Она добавлена к сравнению как эталонный способ устранения искажений из предположения, что с ней результаты стереосопоставления будут 
самыми лучшими. Использование её на реальных объективах ограничено отсутствием возможностей по калибровке. 

Кроме того, виртуальная сцена позволяет разместить сразу несколько объектов в одной точке пространства, таким образом возможно
в существующую модель в Unity добавить ещё 2 камеры, совпадающие по параметрам и положению с виртуальными камерами на 
рисунке \ref{pic:2cam_scheme}. Разрешение новых камер выбрано исходя из размеров проекции $\nu_i$ области интереса на широкоугольном
снимке. Для камер "рыбий глаз" с разрешением $1080*1080$ пикселей она составляет 287482 пикселей, что аналогично камере 
с разрешением $540*540$. Камеры полностью подчиняются перспективной проекции (\ref{eq:uv_to_xyz}) и не содержат каких либо оптических искажений.
 Далее эти камеры будут называться эталонными камерами, а стереопара, которую они составляют - эталонной стереопарой (обозначена на графиках как REG).

Наиболее сильные искажения на изображении "рыбий глаз" возникают по краям изображения, поэтому исследуемая виртуальная стереопара построена так,
чтобы зона пересечения полей зрения камер располагалась на максимальном удалении от центра. Это соответствует ортогональному расположению
камер с углом между принципиальными осями в $90^\circ$.  При этом объективы камер имеют одинаковые параметры, поле зрения в $180^\circ$,
 а расстояние между ними составляет 20см. %Описанная конфигурация изображена на рисунке \ref{pic:2cam_scheme}.

% Здесь область пересечения полей зрения камер $C_0$ и $C_1$ обозначенная красным.
% Эта область эквивалентна области пересечения полей зрения двух камер с полями зрения $90^\circ$ (обозначены
% оранжевым), повёрнутых на $\beta_0 = \beta_1  = 45^\circ$ в сторону области интереса. Тогда $B$ - база стереопары.
